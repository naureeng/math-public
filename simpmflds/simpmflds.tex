
\documentclass{amsart}

\usepackage[colorlinks=true]{hyperref}
\usepackage{enumerate}
\usepackage{color}
\usepackage{tikz-cd}
\usepackage{amssymb}
% \usepackage{centernot}

\theoremstyle{plain}
\newtheorem{theorem}{Theorem}
\newtheorem{lemma}[theorem]{Lemma}
\newtheorem{proposition}[theorem]{Proposition}
\newtheorem{corollary}[theorem]{Corollary}

\theoremstyle{definition}
\newtheorem{definition}[theorem]{Definition}
\newtheorem{example}[theorem]{Example}
\newtheorem{exercise}[theorem]{Exercise}

\theoremstyle{remark}
\newtheorem{remark}[theorem]{Remark}
\newtheorem{question}[theorem]{Question}

% Fonts
\newcommand{\A}{\mathbb{A}}
\newcommand{\C}{\mathbb{C}}
\newcommand{\F}{\mathbb{F}}
\newcommand{\R}{\mathbb{R}}
\newcommand{\Q}{\mathbb{Q}}
\newcommand{\Z}{\mathbb{Z}}
\newcommand{\N}{\mathbb{N}}
\newcommand{\G}{\mathbb{G}}
\newcommand{\fr}{\mathfrak}

% Topology/geometry

\DeclareMathOperator{\Gr}{Gr}
\DeclareMathOperator{\Fl}{Fl}
\DeclareMathOperator{\PP}{\mathbb{P}}
\DeclareMathOperator{\Der}{Der}
\DeclareMathOperator{\Lie}{Lie}
\DeclareMathOperator{\SL}{SL}
\DeclareMathOperator{\GL}{GL}
\DeclareMathOperator{\SO}{SO}
\DeclareMathOperator{\UU}{U}
\DeclareMathOperator{\OO}{O}
\DeclareMathOperator{\Sp}{Sp}
\DeclareMathOperator{\HH}{H}
\DeclareMathOperator{\Symp}{Symp}
\DeclareMathOperator{\Spin}{Spin}
\DeclareMathOperator{\Pin}{Pin}
\DeclareMathOperator{\Td}{Td}
\DeclareMathOperator{\ind}{ind}
\DeclareMathOperator{\vect}{Vect}
\DeclareMathOperator{\Op}{Op}
\DeclareMathOperator{\ev}{ev}

% Representation theory

\DeclareMathOperator{\Ad}{Ad}
\DeclareMathOperator{\tr}{tr}
\DeclareMathOperator{\Str}{str}

% Algebra

\DeclareMathOperator{\End}{End}
\DeclareMathOperator{\Aut}{Aut}
\DeclareMathOperator{\Hom}{Hom}
\DeclareMathOperator{\sHom}{\mathscr{H}\!om}
\DeclareMathOperator{\sEnd}{\mathscr{E}\!nd}
\DeclareMathOperator{\id}{id}
\DeclareMathOperator{\irr}{irr}
\DeclareMathOperator{\Diff}{Diff}
\DeclareMathOperator{\gr}{gr}
\DeclareMathOperator{\im}{im}
\DeclareMathOperator{\ad}{ad}
\DeclareMathOperator{\rk}{rk}
\DeclareMathOperator{\Spec}{Spec}
\DeclareMathOperator{\RSpec}{RSpec}
\DeclareMathOperator{\Specm}{Specm}
\DeclareMathOperator{\Stab}{Stab}
\DeclareMathOperator{\Sym}{Sym}
\DeclareMathOperator{\Ext}{Ext}
\DeclareMathOperator{\ch}{ch}
\DeclareMathOperator{\cone}{cone}
\DeclareMathOperator{\cl}{Cl}
\DeclareMathOperator{\coker}{coker}

% Category theory

\DeclareMathOperator*{\colim}{colim}
\DeclareMathOperator*{\Map}{Map}
\DeclareMathOperator*{\Maps}{Maps}

%\makeatletter
%\renewcommand\d[1]{\mspace{6mu}\mathrm{d}#1\@ifnextchar\d{\mspace{-3mu}}{}}
%\makeatother
%
%\newcommand{\naturalto}{%
%    \mathrel{\vbox{\offinterlineskip
%        \mathsurround=0pt
%        \ialign{\hfil##\hfil\cr
%            \normalfont\scalebox{1.2}{.}\cr
%                            %      \noalign{\kern-.05ex}
%        $\longrightarrow$\cr}
%    }}%
%}



\DeclareMathOperator{\DC}{DC} % differential characters
\DeclareMathOperator{\Del}{Del} % deligne cohomology
% \newcommand{\fsl}[1]{{\centernot{#1}}}
\renewcommand\d{\mathsf{D}}
\DeclareMathOperator{\Fun}{Fun}
\DeclareMathOperator{\Sing}{Sing}
\DeclareMathOperator{\Ob}{Ob}
\newcommand{\op}{\text{op}}

\DeclareMathOperator{\Tot}{Tot}
\DeclareMathOperator{\Nat}{Nat}
\DeclareMathOperator{\pr}{pr}
\DeclareMathOperator{\wt}{wt}
\DeclareMathOperator{\holim}{holim}
\DeclareMathOperator{\hocolim}{hocolim}
\newcommand{\desc}{\text{desc}}
\newcommand{\h}{\text{h}}
\newcommand{\pt}{\text{pt}}
\newcommand{\cyc}{\text{cyc}}


\DeclareMathOperator{\srep}{srep}
\DeclareMathOperator{\Emb}{Emb}




\usepackage{marginnote}

\title{Differential forms on $B_\bullet G$}

\author{Nilay Kumar}

\date{May 2019}

\begin{document}

\begin{abstract}
    We begin by recalling basic notions around differential forms on simplicial
    manifolds. The main object of study is the complex $A^q(B_\bullet G)$ of
    $q$-forms on the classifying space of a compact Lie group $G$. We compute
    the cohomology of this complex, originally due to Bott, and discuss as time
    permits some applications and extensions.
    These notes were prepared for John Francis' seminar course in the spring 
    quarter of 2019 at Northwestern.
\end{abstract}

\maketitle

\section{Simplicial manifolds}

Write $\Delta$ for the simplicial indexing category and $\mathsf{Mfld}$ for the 
category of finite-dimensional smooth manifolds (without boundary).

\begin{definition}
    A simplicial manifold $X_\bullet$ is a functor $X: \Delta^\op\to 
    \mathsf{Mfld}$. A map of simplicial manifolds is a natural transformation of 
    functors, and the resulting category is denoted $\mathsf{sMfld}$.
\end{definition}

\begin{example}
    Here are a few standard examples of simplicial manifolds:\marginnote{finish}
    \begin{enumerate}
        \item discrete manifolds
        \item classifying space of a lie group, universal space
        \item nerve of a lie groupoid, Cech nerve, prev example
        \item $W, \bar W$ of simplicial Lie group
        \item Getzler-Pohorence, integration over local Lagrangians
    \end{enumerate}
\end{example}

Let me give some motivation for the introduction of simplicial manifolds.
As simplicial manifolds are in particular simplicial topological spaces, it 
is not surprising that the novelty in studying simplicial manifolds comes from 
applications where the smooth structure is of interest.

Consider, for example, the Chern-Weil approach to characteristic classes. Given 
a complex vector bundle $\pi: E\to M$ one obtains the Chern classes of $E$ by 
choosing a connection $\nabla$ on $E$ and then taking the trace of a certain 
polynomial in the curvature $F^\nabla\in\Omega^2(M)$. Changing the connection 
modifies the resulting form by an exact form (the differential of an expression 
involving the Chern-Simons secondary invariant) whence the de Rham cohomology 
class is independent of this choice. From a topological perspective, this 
cohomology class should be pulled back from an appropriate classifying space 
along a map classifying the vector bundle. One would like to lift this to
a chain level statement but the classifying space is not a manifold in general
and thus does not obviously support differential forms, let alone universal 
Chern-Weil forms. Such a lift to the de Rham complex was first constructed by 
Shulman in his Berkeley thesis, and roughly imitates the Chern-Weil construction 
for the simplicial $G$-bundle $E_\bullet G\to B_\bullet G$.

Another motivation comes from physics. In many quantum field theories the 
numerical invariants calculated by path integral methods fail to make sense, 
even at a physical level of rigor, due to the presence of what are called 
anomalies. Roughly speaking the presence of an anomaly signifies that the 
integrand is not a function but instead a section of a non-trivial line bundle. 
This was first made precise in an example by Quillen and then generalized by
Bismut and Freed. Often the anomaly can be ``trivialized'' by further geometric 
constraints on the underlying spacetime manifold such as a spin structure.
For some string theories the anomaly is trivialized by a string structure on 
spacetime. The string group is a 3-connected cover of the spin group
but is a priori defined only as a topological group. Promoting the string 
group to some sort of smooth object is not so easy; if one wants to stay in the 
world of finite-dimensional manifolds, one is forced to view it as a 2-group: a 
simplicial manifold satisfying certain horn-lifting properties. Much more 
generally, it is interesting to ask for an analog of Lie's third theorem for 
$L_\infty$-algebras. The ``Lie group'' corresponding to a given 
$L_\infty$-algebra is naturally a simplicial Banach manifold.

Finally, as less of a motivation and more of a categorical outlook, I would like 
to mention that Kan simplicial manifolds offer relatively concrete and hands-on
representatives for (nice enough) smooth stacks. Two Kan simplicial 
manifolds are to be considered equivalent as smooth stacks if they are connected 
by ``hypercovers.'' These are maps of simplicial manifolds that generalize the 
augmentation of the Cech nerve of a good open cover. That such maps should be 
equivalences is already evident in the definition of a smooth manifold via 
atlases.

With these remarks in mind, let us turn to the notion of differential forms on 
simplicial manifolds. 
\begin{definition}
    Let $X_\bullet$ be a simplicial manifold. The de Rham complex of $X_\bullet$ 
    is the cosimplicial cochain complex given as the composition
    \begin{equation*}
        A^*X : \Delta \xrightarrow{X_\bullet} \mathsf{Mfld}^\op
        \xrightarrow{A^*(-)} \mathsf{Ch}(\mathsf{Vect}_\R).
    \end{equation*}
    For a fixed $q\geq 0$ in particular we obtain a cosimplicial vector space of
    $q$-forms on $X$,
    \begin{equation*}
        A^qX : \Delta \to \mathsf{Vect}_\R.
    \end{equation*}
\end{definition}

As usual there is a functor
\begin{equation*}
    \text{Moore}: \mathsf{cCh(Vect_\R)} \to \mathsf{Ch(Ch(Vect_\R))}
\end{equation*}
that assigns to a cosimplicial cochain complex the corresponding Moore double 
complex (one might instead take the normalized double complex). Now to a double 
complex we can assign its total complex: we will often abuse terminology and 
refer to this complex as the de Rham complex of $X_\bullet$.

\begin{example}
    Here are the two most basic examples:
    \begin{itemize}
        \item If $X_\bullet$ is zero-dimensional, i.e. just a simplicial set,
            then the 
            de Rham complex $A^*X$ is precisely the simplicial cochain complex
            of $X_\bullet$ with coefficients in $\R$.
        \item If $X_\bullet=X$ is the constant simplicial object on $X\in 
            \mathsf{Mfld}$ then $A^*X$ is (quasi-isomorphic to, unless one took
            the normalized complex) the usual complex of differential forms on
            $X$.
    \end{itemize}
\end{example}

\begin{remark}
    There is a product on the de Rham complex of a simplicial manifold
    coming from the wedge product of differential forms that 
    turns out not to be commutative on the nose -- instead, it gives the de
    Rham complex the structure of a $C_\infty$-algebra. The homotopy coherent 
    nature of the multiplication is not surprising given the first example 
    above.
\end{remark}

As we noted earlier, a simplicial manifold is naturally a simplicial topological 
space. Thus we obtain a functor
\begin{equation*}
    |-| : \mathsf{sMfld} \to \mathsf{Spaces}
\end{equation*}
given by first taking the underlying simplicial space and then taking the 
geometric realization.

de Rham complex of a simplicial manifold, simplicial de Rham theorem

\section{Bott's argument}

goal: compute cohomology of the vertical complex of forms on BG

3 steps: basic on EG, functions on EG, Dold-Puppe argument

basic forms on on EG: introduce the suspension cosimplicial object

use Haar measure on G to show functions on EG are (invariantly) contractible

Dold-Puppe argument to pass the wedge power inside as a symmetric power

\section{Applications}

notion of closed forms, invariance under hypercovers

symplectic structures on BG

application: symplectic reduction, quasi-Hamiltonian reduction, Chern-Simons 
theory

Shulman's construction of the symplectic form

unique differential refinement of chern classes

extension to n-gerbes

\end{document}

