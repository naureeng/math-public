
\documentclass{amsart}

\usepackage[colorlinks=true]{hyperref}
\usepackage{enumerate}
\usepackage{color}
\usepackage{tikz-cd}
\usepackage{amssymb}
% \usepackage{centernot}

\theoremstyle{plain}
\newtheorem{theorem}{Theorem}
\newtheorem{lemma}[theorem]{Lemma}
\newtheorem{proposition}[theorem]{Proposition}
\newtheorem{corollary}[theorem]{Corollary}

\theoremstyle{definition}
\newtheorem{definition}[theorem]{Definition}
\newtheorem{example}[theorem]{Example}
\newtheorem{exercise}[theorem]{Exercise}

\theoremstyle{remark}
\newtheorem{remark}[theorem]{Remark}
\newtheorem{question}[theorem]{Question}

% Fonts
\newcommand{\A}{\mathbb{A}}
\newcommand{\C}{\mathbb{C}}
\newcommand{\F}{\mathbb{F}}
\newcommand{\R}{\mathbb{R}}
\newcommand{\Q}{\mathbb{Q}}
\newcommand{\Z}{\mathbb{Z}}
\newcommand{\N}{\mathbb{N}}
\newcommand{\G}{\mathbb{G}}
\newcommand{\fr}{\mathfrak}

% Topology/geometry

\DeclareMathOperator{\Gr}{Gr}
\DeclareMathOperator{\Fl}{Fl}
\DeclareMathOperator{\PP}{\mathbb{P}}
\DeclareMathOperator{\Der}{Der}
\DeclareMathOperator{\Lie}{Lie}
\DeclareMathOperator{\SL}{SL}
\DeclareMathOperator{\GL}{GL}
\DeclareMathOperator{\SO}{SO}
\DeclareMathOperator{\UU}{U}
\DeclareMathOperator{\OO}{O}
\DeclareMathOperator{\Sp}{Sp}
\DeclareMathOperator{\HH}{H}
\DeclareMathOperator{\Symp}{Symp}
\DeclareMathOperator{\Spin}{Spin}
\DeclareMathOperator{\Pin}{Pin}
\DeclareMathOperator{\Td}{Td}
\DeclareMathOperator{\ind}{ind}
\DeclareMathOperator{\vect}{Vect}
\DeclareMathOperator{\Op}{Op}
\DeclareMathOperator{\ev}{ev}

% Representation theory

\DeclareMathOperator{\Ad}{Ad}
\DeclareMathOperator{\tr}{tr}
\DeclareMathOperator{\Str}{str}

% Algebra

\DeclareMathOperator{\End}{End}
\DeclareMathOperator{\Aut}{Aut}
\DeclareMathOperator{\Hom}{Hom}
\DeclareMathOperator{\sHom}{\mathscr{H}\!om}
\DeclareMathOperator{\sEnd}{\mathscr{E}\!nd}
\DeclareMathOperator{\id}{id}
\DeclareMathOperator{\irr}{irr}
\DeclareMathOperator{\Diff}{Diff}
\DeclareMathOperator{\gr}{gr}
\DeclareMathOperator{\im}{im}
\DeclareMathOperator{\ad}{ad}
\DeclareMathOperator{\rk}{rk}
\DeclareMathOperator{\Spec}{Spec}
\DeclareMathOperator{\RSpec}{RSpec}
\DeclareMathOperator{\Specm}{Specm}
\DeclareMathOperator{\Stab}{Stab}
\DeclareMathOperator{\Sym}{Sym}
\DeclareMathOperator{\Ext}{Ext}
\DeclareMathOperator{\ch}{ch}
\DeclareMathOperator{\cone}{cone}
\DeclareMathOperator{\cl}{Cl}
\DeclareMathOperator{\coker}{coker}

% Category theory

\DeclareMathOperator*{\colim}{colim}
\DeclareMathOperator*{\Map}{Map}
\DeclareMathOperator*{\Maps}{Maps}

%\makeatletter
%\renewcommand\d[1]{\mspace{6mu}\mathrm{d}#1\@ifnextchar\d{\mspace{-3mu}}{}}
%\makeatother
%
%\newcommand{\naturalto}{%
%    \mathrel{\vbox{\offinterlineskip
%        \mathsurround=0pt
%        \ialign{\hfil##\hfil\cr
%            \normalfont\scalebox{1.2}{.}\cr
%                            %      \noalign{\kern-.05ex}
%        $\longrightarrow$\cr}
%    }}%
%}



\DeclareMathOperator{\DC}{DC} % differential characters
\DeclareMathOperator{\Del}{Del} % deligne cohomology
% \newcommand{\fsl}[1]{{\centernot{#1}}}
\renewcommand\d{\mathsf{D}}
\DeclareMathOperator{\Fun}{Fun}
\DeclareMathOperator{\Sing}{Sing}
\DeclareMathOperator{\Ob}{Ob}
\newcommand{\op}{\text{op}}

\DeclareMathOperator{\Tot}{Tot}
\DeclareMathOperator{\Nat}{Nat}
\DeclareMathOperator{\pr}{pr}
\DeclareMathOperator{\wt}{wt}
\DeclareMathOperator{\holim}{holim}
\DeclareMathOperator{\hocolim}{hocolim}
\newcommand{\desc}{\text{desc}}
\newcommand{\h}{\text{h}}
\newcommand{\pt}{\text{pt}}
\newcommand{\cyc}{\text{cyc}}


\DeclareMathOperator{\srep}{srep}
\DeclareMathOperator{\Emb}{Emb}




\DeclareMathOperator{\diag}{\text{diag}}

\usepackage{marginnote}

\title{Differential forms on $B_\bullet G$}

\author{Nilay Kumar}

\date{May 2019}

\begin{document}

\begin{abstract}
    We begin by recalling basic notions around differential forms on simplicial
    manifolds. The main object of study is the complex $A^q(B_\bullet G)$ of
    $q$-forms on the classifying space of a compact Lie group $G$. We compute
    the cohomology of this complex, originally due to Bott, and discuss as time
    permits some applications and extensions.
    These notes were prepared for John Francis' seminar course in the spring 
    quarter of 2019 at Northwestern.
\end{abstract}

\maketitle

\section{Simplicial manifolds}

Write $\Delta$ for the simplicial indexing category and $\mathsf{Mfld}$ for the 
category of finite-dimensional smooth manifolds (without boundary).

\begin{definition}
    A simplicial manifold $X_\bullet$ is a functor $X: \Delta^\op\to 
    \mathsf{Mfld}$. A map of simplicial manifolds is a natural transformation of 
    functors, and the resulting category is denoted $\mathsf{sMfld}$.
\end{definition}

\begin{example}
    Here are a few standard examples of simplicial manifolds:\marginnote{finish}
    \begin{enumerate}
        \item discrete manifolds
        \item classifying space of a lie group, universal space
        \item nerve of a lie groupoid, Cech nerve, prev example
        \item $W, \bar W$ of simplicial Lie group
        \item Getzler-Pohorence, integration over local Lagrangians
    \end{enumerate}
\end{example}

Let me give some motivation for the introduction of simplicial manifolds.
As simplicial manifolds are in particular simplicial topological spaces, it 
is not surprising that the novelty in studying simplicial manifolds comes from 
applications where the smooth structure is of interest.

Consider, for example, the Chern-Weil approach to characteristic classes. Given 
a complex vector bundle $\pi: E\to M$ one obtains the Chern classes of $E$ by 
choosing a connection $\nabla$ on $E$ and then taking the trace of a certain 
polynomial in the curvature $F^\nabla\in\Omega^2(M)$. Changing the connection 
modifies the resulting form by an exact form (the differential of an expression 
involving the Chern-Simons secondary invariant) whence the de Rham cohomology 
class is independent of this choice. From a topological perspective, this 
cohomology class should be pulled back from an appropriate classifying space 
along a map classifying the vector bundle. One would like to lift this to
a chain level statement but the classifying space is not a manifold in general
and thus does not obviously support differential forms, let alone universal 
Chern-Weil forms. Such a lift to the de Rham complex was first constructed by 
Shulman in his Berkeley thesis, and roughly imitates the Chern-Weil construction 
for the simplicial $G$-bundle $E_\bullet G\to B_\bullet G$.

Another motivation comes from physics. In many quantum field theories the 
numerical invariants calculated by path integral methods fail to make sense, 
even at a physical level of rigor, due to the presence of what are called 
anomalies. Roughly speaking the presence of an anomaly signifies that the 
integrand is not a function but instead a section of a non-trivial line bundle. 
This was first made precise in an example by Quillen and then generalized by
Bismut and Freed. Often the anomaly can be ``trivialized'' by further geometric 
constraints on the underlying spacetime manifold such as a spin structure.
For some string theories the anomaly is trivialized by a string structure on 
spacetime. The string group is a 3-connected cover of the spin group
but is a priori defined only as a topological group. Promoting the string 
group to some sort of smooth object is not so easy; if one wants to stay in the 
world of finite-dimensional manifolds, one is forced to view it as a 2-group: a 
simplicial manifold satisfying certain horn-lifting properties. Much more 
generally, it is interesting to ask for an analog of Lie's third theorem for 
$L_\infty$-algebras. The ``Lie group'' corresponding to a given 
$L_\infty$-algebra is naturally a simplicial Banach manifold.

Finally, as less of a motivation and more of a categorical outlook, I would like 
to mention that Kan simplicial manifolds offer relatively concrete and hands-on
representatives for (nice enough) smooth stacks. Two Kan simplicial 
manifolds are to be considered equivalent as smooth stacks if they are connected 
by ``hypercovers.'' These are maps of simplicial manifolds that generalize the 
augmentation of the Cech nerve of a good open cover. That such maps should be 
equivalences is already evident in the definition of a smooth manifold via 
atlases.

With these remarks in mind, let us turn to the notion of differential forms on 
simplicial manifolds. 
\begin{definition}
    Let $X_\bullet$ be a simplicial manifold. The de Rham complex of $X_\bullet$ 
    is the cosimplicial cochain complex given as the composition
    \begin{equation*}
        A^*X : \Delta \xrightarrow{X_\bullet} \mathsf{Mfld}^\op
        \xrightarrow{A^*(-)} \mathsf{Ch}(\mathsf{Vect}_\R).
    \end{equation*}
    For a fixed $q\geq 0$ in particular we obtain a cosimplicial vector space of
    $q$-forms on $X$,
    \begin{equation*}
        A^qX : \Delta \to \mathsf{Vect}_\R.
    \end{equation*}
\end{definition}

As usual there is a functor
\begin{equation*}
    N: \mathsf{cCh(Vect_\R)} \to \mathsf{Ch(Ch(Vect_\R))}
\end{equation*}
that assigns to a cosimplicial cochain complex the corresponding normalized
double complex. Now to a double 
complex we can assign its total complex: we will often abuse terminology and 
refer to this complex as the de Rham complex of $X_\bullet$.

\begin{example}
    Here are the two most basic examples:
    \begin{itemize}
        \item If $X_\bullet$ is zero-dimensional, i.e. just a simplicial set,
            then the 
            de Rham complex $A^*X$ is precisely the simplicial cochain complex
            of $X_\bullet$ with coefficients in $\R$.
        \item If $X_\bullet=X$ is the constant simplicial object on $X\in 
            \mathsf{Mfld}$ then $A^*X$ is (quasi-isomorphic to, unless one took
            the normalized complex) the usual complex of differential forms on
            $X$.
    \end{itemize}
\end{example}

\begin{remark}
    There is a product on the de Rham complex of a simplicial manifold
    coming from the wedge product of differential forms that 
    turns out not to be commutative on the nose -- instead, it gives the de
    Rham complex the structure of a $C_\infty$-algebra. The homotopy coherent 
    nature of the multiplication is not surprising given the first example 
    above.
\end{remark}

As we noted earlier, a simplicial manifold is naturally a simplicial topological 
space. Thus we obtain a functor
\begin{equation*}
    |-| : \mathsf{sMfld} \to \mathsf{Spaces}
\end{equation*}
given by first taking the underlying simplicial space and then taking the 
geometric realization.

simplicial de Rham theorem \marginnote{finish}

\section{Bott's argument}

In this section we focus on $q$-forms on the simplicial manifold $B_\bullet G$.
Recall that $A^q(B_\bullet G)$ is a cosimplicial vector space to which we
assign the associated normalized complex. Following Bott, we will compute the 
cohomology of this complex; I thank Ezra Getzler for explaining this proof
to me.

\begin{theorem}[Bott '73]
    Let $G$ be a compact Lie group and $q$ be a non-negative integer.
    Then
    \begin{equation*}
        H^i(A^q(B_\bullet G)) \cong
        \begin{cases}
            \Sym^q(\fr g^\vee)^G & i=q \\
            0 & i\neq q
        \end{cases}
    \end{equation*}
\end{theorem}

The proof of this result has three ingredients:
\begin{enumerate}
    \item $A^q(B_\bullet G)$ can be obtained as the \textit{basic} 
        $q$-forms on the universal principal $G$-bundle $E_\bullet G$
    \item by compactness of $G$, the cosimplicial $G$-module $C^\infty(E_\bullet 
        G)$ admits an extra codegeneracy \marginnote{as a simplicial 
        $G$-module??} 
    \item the first (nonadditive) derived functor of the
        exterior power functor $\Lambda^q$ is a $q$-fold shift of the symmetric
        power functor $\Sym^q$.
\end{enumerate}

Let's start with the first step. Instead of working with differential forms on 
the classifying space $B_\bullet G$, we will work on the total space, which 
affords greater flexibility.

\begin{proposition}
    Let $\pi: P\to M$ be a smooth principal $G$-bundle. Then the pullback 
    \begin{equation*}
        A^q(M) \xrightarrow{\pi^*} A^q(P)
    \end{equation*}
    induces an isomorphism of $A^q(M)$ onto the subspace $A^q_\text{basic}(P)$ 
    of basic $q$-forms on $P$. Recall that $\omega\in A^q(P)$ is basic if it is 
    $G$-invariant and horizontal:
    \begin{equation*}
        R_g^*\omega = \omega \qquad \text{and} \qquad \omega|_{\pi^{-1}(m)}=0
    \end{equation*}
    for all $m\in M$. Here $R_g: P\to P$ is the right action of $g\in G$ on $P$.
\end{proposition}
The proof is straightforward.

\begin{corollary}
    Pullback along $\pi: E_\bullet G \to B_\bullet G$ induces an isomorphism
    of complexes
    \begin{equation*}
        A^q(B_\bullet G) \cong A^q_\text{basic}(E_\bullet G) \subset 
        A^q(E_\bullet G).
    \end{equation*}
\end{corollary}

\begin{definition}
    Define the cosimplicial vector space $C^\bullet\R$ as the composite
    \begin{equation*}
        C^\bullet\R : \Delta \hookrightarrow \mathsf{Set} 
        \xrightarrow{\mathsf{free}} \mathsf{Vect}_\R.
    \end{equation*}
    In other words, $C^n\R$ is the vector space of dimension $n+1$: denote by  
    $e_i$ the basis vector of $C^n\R=\R^{n+1}$ corresponding to $i\in [n]$. The 
    $i$th coface map is the inclusion that misses $e_i$ and the $i$th 
    codegeneracy map is the surjection that sends $e_i$ and $e_{i+1}$ to $e_i$.
    There is a (levelwise) surjection of cosimplicial vector spaces
    \begin{equation*}
        C^\bullet \R \xrightarrow{s} \R \to 0
    \end{equation*}
    to the constant one-dimensional cosimplicial vector space given (in degree 
    $n$)
    \begin{equation*}
        s\left( \sum_{i=0}^{n+1} a_i e_i \right) = \sum_{i=0}^{n+1} a_i.
    \end{equation*}
    If we define $\Sigma^\bullet \R=\ker s$ we obtain a short exact sequence of 
    cosimplicial vector spaces
    \begin{equation*}
        0 \to \Sigma^\bullet \R \to C^\bullet \R \to \R \to 0.
    \end{equation*}
\end{definition}

\begin{remark}
    Beware that the map $s$ is not a coaugmentation of $C^\bullet\R$ -- it
    goes the wrong way!
\end{remark}

Now for any cosimplicial vector space $V^\bullet$ we obtain cosimplicial vector 
spaces
\begin{align*}
    C^\bullet V &= C^\bullet\R \otimes_\R V^\bullet \\
    \Sigma^\bullet V &= \Sigma^\bullet \R \otimes_\R V^\bullet
\end{align*}
that we call the cone and suspension of $V^\bullet$, respectively.
They fit into a short exact sequence
\begin{equation*}
    0 \to \Sigma^\bullet V \to C^\bullet V \to V^\bullet \to 0.
\end{equation*}
Notice that \marginnote{Eilenberg-Zilber?}
\begin{align*}
    H^*NC^\bullet V &= H^*(N(C^\bullet\R \otimes V^\bullet)) \\
    &= H^*NC^\bullet\R \otimes H^*NV^\bullet \\
    &= 0
\end{align*}
whence
\begin{equation*}
    H^*NC^\bullet V = 0 \qquad \text{and} \qquad H^*N\Sigma^\bullet V=V[-1].
\end{equation*}

With this notation established, we arrive at the statement of the first result.
\begin{lemma}[Decomposition lemma]
    Let $G$ be a Lie group and let $\fr g^\vee$ be the vector space of 
    left-invariant 1-forms on $G$, viewed as a $G$-module under 
    right-multiplication. Then
    \begin{equation*}
        A^q(B_\bullet G) \cong \diag\left( A^0(E_\bullet G) \otimes 
        \Lambda^q\Sigma^\bullet\fr g^\vee\right)^G.
    \end{equation*}
\end{lemma}
In words: the cosimplicial vector space of $q$-forms on $B_\bullet G$ is 
isomorphic to the diagonal cosimplicial vector space obtained from the 
$G$-invariants of the bicosimplicial $G$-module in parentheses.
\begin{proof}
    As noted above,
    \begin{equation*}
        A^q(B_\bullet G) \cong A^q_\text{basic}(E_\bullet G).
    \end{equation*}
    The right hand side of the decomposition above is just another way of 
    capturing the basic forms on $E_\bullet G$. We can see this as follows.
    If we choose a basis for the Lie algebra $\fr g$ then we obtain a global 
    frame for the cotangent bundle of $G$, whence
    $A^1(G^{n+1}) = A^0(G^{n+1})\otimes(\fr g^\vee)^{\oplus n+1}.$
    More explicitly, we have an isomorphism
    \begin{align*}
        \phi_n : A^0(E_nG) \otimes C^n\fr g^\vee &\to A^1(E_nG) \\
        \sum_{k=0}^n f_k \otimes \xi_k &\mapsto \sum_{k=0}^n f_k \cdot \pi_k^* 
        \xi_k,
    \end{align*}
    where $\pi_k:E_nG=G^{n+1}\to G$ is projection onto the $k$th factor.
    We are interested in the subspace of horizontal forms, i.e. those that 
    restricted to a fiber of $E_nG \to B_nG$ are zero. For $p=(p_0,\ldots, 
    p_n)\in E_nG$ write the inclusion of a fiber
    \begin{align*}
        \iota_p : G &\to E_nG=G^{n+1} \\
        g &\mapsto (p_0g, \ldots, p_ng).
    \end{align*}
    If we now restrict along $\iota_p$ we get
    \begin{align*}
        \iota_p^*\left( \sum_{k=0}^n f_k \cdot \pi_k^*\xi_k \right)
        &= \sum_{k=0}^n f_k \cdot (\pi_k\circ \iota_p)^* \xi_k
        = \sum_{k=0}^n f_k \cdot \xi_k.
    \end{align*}
    Here we have used that the composition $\pi_k\circ \iota_p: G\to G$ is
    left-multiplication by $p_k$,
    \begin{equation*}
        \pi_k(\iota_p(g)) = \pi_k(p_0g,\ldots,p_ng) = p_kg,
    \end{equation*}
    together with the left-invariance of $\xi_k\in\fr g^\vee \subset A^1(G)$.
    We thus conclude that
    \begin{equation*}
        A^1_\text{horiz}(E_nG) = A^0(E_nG) \otimes \Sigma^n\fr g^\vee,
    \end{equation*}
    from which we obtain
    \begin{equation*}
        A^q_\text{horiz}(E_nG) = A^0(E_nG) \otimes \Lambda^q\Sigma^n\fr g^\vee,
    \end{equation*}
    Taking invariants (and checking cosimplicial structure maps) results in the 
    desired decomposition
    \begin{equation*}
        A^q_\text{basic}(E_\bullet G) = \diag \left( A^0(E_\bullet G)\otimes 
        \Lambda^q \Sigma^\bullet\fr g^\vee \right)^G.
    \end{equation*}
\end{proof}

\begin{proposition}
    For $G$ compact the Haar measure furnishes a homotopy equivalence
    \begin{equation*}
        \diag\left( A^0(E_\bullet G) \otimes \Lambda^q\Sigma^\bullet\fr 
        g^\vee \right)^G \simeq \left( \Lambda^q\Sigma^\bullet\fr g^\vee 
        \right)^G.
    \end{equation*}
    of cosimplicial vector spaces.\marginnote{state this correctly}
\end{proposition}
\begin{proof}
    \marginnote{Finish}
\end{proof}

We now have an equivalence of cosimplicial vector spaces
\begin{equation*}
    A^q(B_\bullet G) \simeq \left( \Lambda^q\Sigma^\bullet\fr g^\vee \right)^G.
\end{equation*}
To finish the proof of Bott's result we must now compute the cohomology of the 
normalized complex of the right-hand side.
\begin{proposition}
    There is a weak equivalence\marginnote{State this correctly}
    \begin{equation*}
        N\left( \Lambda^q\Sigma^\bullet\fr g^\vee \right)^G
        \simeq \Sym^q(\fr g^\vee)^G[-q].
    \end{equation*}
\end{proposition}
\begin{proof}
    In the language of Dold and Puppe, we are computing the first (nonadditive) 
    derived functor of the $q$th tensor power functor.\marginnote{finish}
\end{proof}


\section{Applications}

The computation of differential forms on $B_\bullet G$ has a number of 
interesting applications; here I will focus on some immediate corollaries 
pertaining to the topology/geometry of the classifying stack itself.

There is a relatively recent notion of (shifted) symplectic structures on
(derived) stacks due to Pantev-Toen-Vaqui\'e-Vezzosi: one of their fundamental 
examples is the classifying stack of a reductive algebraic group. In the smooth 
setting that we are working in, we obtain the following.
\begin{corollary}
    For $G$ a compact Lie group,
    the set of 2-shifted symplectic structures on $B_\bullet G$ is the 
    set of Killing forms on $\fr g$ (nondegenerate invariant symmetric
    bilinear forms on $\fr g$).
\end{corollary}

Once we define what a 2-shifted closed form on a simplicial manifold is, this 
corollary is immediate. That the notion needs defining in the first place stems
from the fact a cocycle in a complex is not a homotopy 
invariant object. The way out is to notice that the de Rham complex of 
manifold comes with a canonical filtration known as the Hodge filtration
\begin{equation*}
    F^pA^*X = 0 \to A^pX \to A^{p+1}X \to \cdots,
\end{equation*}
and that
\begin{equation*}
    H^0(F^pA^*X[p]) = A^p_\text{cl}X.
\end{equation*}

\begin{definition}
    Let $X_\bullet$ be a simplicial manifold. The complex of closed $p$-forms on
    $X_\bullet$ is defined to be the total complex of the double complex 
    obtained by applying $F^pA^*$ to $X_\bullet$,
    \begin{equation*}
        A_\text{cl}^p(X_\bullet) = F^pA^*X_\bullet[p].
    \end{equation*}
    An $n$-shifted closed $p$-form $\omega$ is a degree $n$ cocycle of this
    complex:
    \begin{equation*}
        \omega \in Z^{n+p}(F^pA^*X_\bullet).
    \end{equation*}
\end{definition}

Let's unpack what an $n$-shifted closed $p$-form is, more concretely.
We have
\begin{equation*}
    \omega = (\omega_0, \ldots, \omega_n), \qquad \omega_i\in 
    A^{p+n-i}(X_i)
\end{equation*}
such that
\begin{align*}
    \delta\omega_n &= 0 \\
    \delta\omega_{n-1} &= \pm d\omega_n \\
    \vdots & \\
    \delta\omega_0 &= \pm d\omega_1 \\
    d\omega_0 &=0.
\end{align*}
In other words, a $n$-shifted closed $p$-form is a $\delta$-closed $p$-form
$\omega_n$ on $X_n$ (hence the $n$-shifted $p$-form) whose de Rham differential 
is $\delta$-exact via $\omega_{n-1}\in A^{p+1}(X_{n-1})$. Similarly 
$\omega_{n-1}$ is closed up to a $\delta$-exact form, and so on and so forth 
until $\omega_0$, which is a closed $(p+n)$-form on $X_0$.

\begin{example}
    For $X_\bullet = B_\bullet G$ in particular we have
    \begin{equation*}
        A^p_\text{cl}(B_\bullet G) = 0 \to A^p(G) \to A^p(G^2)\oplus 
        A^{p+1}G \to A^p(G^3)\oplus A^{p+1}(G^2)\oplus A^{p+2}G \to \cdots
    \end{equation*}
    A 2-shifted closed 2-form on $B_\bullet G$ is thus the data of
    \begin{equation*}
        \omega_1\in A^3(G) \qquad \text{and} \qquad \omega_2\in A^2(G^2)
    \end{equation*}
    such that
    \begin{equation*}
        d\omega_1 = 0 \qquad \delta\omega_1 = \pm d\omega_2 \qquad 
        \delta\omega_2=0.
    \end{equation*}
\end{example}

\begin{proposition}
    The cohomology of the complex of closed $p$-forms on the simplicial manifold 
    $B_\bullet G$ is
    \begin{equation*}
        H^q(A^p_\text{cl}B_\bullet G) =
        \begin{cases}
            0 & q < p \\
            \Sym^q(\fr g^\vee)^G & q\geq p
        \end{cases}
    \end{equation*}
\end{proposition}
\begin{proof}
    Consider the spectral sequence for the double complex of closed $p$-forms on 
    $B_\bullet G$ where on the $E_2$-page we first take horizontal cohomology and
    then vertical cohomology. Bott's computation tells us that the $q$th row is 
    $\Sym^q(\fr g^\vee)^G$ concentrated in degree $q$. Thus there are no 
    nontrivial differentials and the spectral sequence collapses at the $E_2$ 
    page. The result follows.
\end{proof}

We see in particular that the set of 2-shifted closed 2-forms
on $B_\bullet G$ is in bijection with 
$G$-invariant symmetric bilinear forms on $\fr g$.
Now, I don't want to go into the notion of nondegeneracy of a 2-form on a 
simplicial manifold, but it is hopefully reasonable that it restricts us to 
nondegenerate such bilinear forms, so I will declare victory here.

\begin{remark}
    Let $G$ be a matrix group like $U(n)$.\marginnote{connectedness?}
    In this case the data of a symplectic structure is the data of forms 
    $\omega_1$ and $\omega_2$ living on products of $U(n)$ and so it is 
    reasonable to hope that, given a Killing form on $\fr{u}(n)$, there is some 
    sort of concrete formula for these differential forms.
    This is the due to Shulman, who proceeds, roughly, by 
    choosing a universal connection on $E_nG\to B_nG$ and using the Killing form 
    on $\fr{u}(n)$ to construct $\omega_1$ and $\omega_2$ following Chern and 
    Weil. In particular, the symplectic form on $B_\bullet U(n)$ is constructed 
    as the Shulman representative for the second Chern class $c_2$.
\end{remark}

\begin{remark}
    We have focused here on $B_\bullet G$, a simplicial representative for the 
    classifying space of principal $G$-bundles. There exist higher categorical 
    analogs of principal bundles known as gerbes: instead of $G$-valued 
    transition functions on double overlaps of a cover, they're defined by 
    transition functions on triple overlaps. At least in the abelian case, there 
    is a nice theory of these ``bundle gerbes'' and they are classified by the 
    higher delooping $B^2A$ (or if you wish $\bar W^2A$).
    It should be straightforward to prove an analog of Bott's result for the
    classifying spaces $B^nA$ for 
    $A$ a compact abelian Lie group ($A=U(1)$ is of interest, say):
    \begin{equation*}
        H^i(A^q(B^n_\bullet A)) \cong
        \begin{cases}
            \Sym^q(\fr a^\vee)^A & q = 2n \\
            0 & q \neq 2n
        \end{cases}
    \end{equation*}
    and to conclude that $B^nA$ is equipped with a $2n$-shifted symplectic 
    structure, though I haven't worked out the details (I expect the proof to be 
    more or less identical if one uses the model $\bar W^nA$).
\end{remark}


\end{document}

