
\documentclass{amsart}

\usepackage[colorlinks=true]{hyperref}
\usepackage{enumerate}
\usepackage{color}
\usepackage{tikz-cd}
\usepackage{amssymb}
% \usepackage{centernot}

\theoremstyle{plain}
\newtheorem{theorem}{Theorem}
\newtheorem{lemma}[theorem]{Lemma}
\newtheorem{proposition}[theorem]{Proposition}
\newtheorem{corollary}[theorem]{Corollary}

\theoremstyle{definition}
\newtheorem{definition}[theorem]{Definition}
\newtheorem{example}[theorem]{Example}
\newtheorem{exercise}[theorem]{Exercise}

\theoremstyle{remark}
\newtheorem{remark}[theorem]{Remark}
\newtheorem{question}[theorem]{Question}

% Fonts
\newcommand{\A}{\mathbb{A}}
\newcommand{\C}{\mathbb{C}}
\newcommand{\F}{\mathbb{F}}
\newcommand{\R}{\mathbb{R}}
\newcommand{\Q}{\mathbb{Q}}
\newcommand{\Z}{\mathbb{Z}}
\newcommand{\N}{\mathbb{N}}
\newcommand{\G}{\mathbb{G}}
\newcommand{\fr}{\mathfrak}

% Topology/geometry

\DeclareMathOperator{\Gr}{Gr}
\DeclareMathOperator{\Fl}{Fl}
\DeclareMathOperator{\PP}{\mathbb{P}}
\DeclareMathOperator{\Der}{Der}
\DeclareMathOperator{\Lie}{Lie}
\DeclareMathOperator{\SL}{SL}
\DeclareMathOperator{\GL}{GL}
\DeclareMathOperator{\SO}{SO}
\DeclareMathOperator{\UU}{U}
\DeclareMathOperator{\OO}{O}
\DeclareMathOperator{\Sp}{Sp}
\DeclareMathOperator{\HH}{H}
\DeclareMathOperator{\Symp}{Symp}
\DeclareMathOperator{\Spin}{Spin}
\DeclareMathOperator{\Pin}{Pin}
\DeclareMathOperator{\Td}{Td}
\DeclareMathOperator{\ind}{ind}
\DeclareMathOperator{\vect}{Vect}
\DeclareMathOperator{\Op}{Op}
\DeclareMathOperator{\ev}{ev}

% Representation theory

\DeclareMathOperator{\Ad}{Ad}
\DeclareMathOperator{\tr}{tr}
\DeclareMathOperator{\Str}{str}

% Algebra

\DeclareMathOperator{\End}{End}
\DeclareMathOperator{\Aut}{Aut}
\DeclareMathOperator{\Hom}{Hom}
\DeclareMathOperator{\sHom}{\mathscr{H}\!om}
\DeclareMathOperator{\sEnd}{\mathscr{E}\!nd}
\DeclareMathOperator{\id}{id}
\DeclareMathOperator{\irr}{irr}
\DeclareMathOperator{\Diff}{Diff}
\DeclareMathOperator{\gr}{gr}
\DeclareMathOperator{\im}{im}
\DeclareMathOperator{\ad}{ad}
\DeclareMathOperator{\rk}{rk}
\DeclareMathOperator{\Spec}{Spec}
\DeclareMathOperator{\RSpec}{RSpec}
\DeclareMathOperator{\Specm}{Specm}
\DeclareMathOperator{\Stab}{Stab}
\DeclareMathOperator{\Sym}{Sym}
\DeclareMathOperator{\Ext}{Ext}
\DeclareMathOperator{\ch}{ch}
\DeclareMathOperator{\cone}{cone}
\DeclareMathOperator{\cl}{Cl}
\DeclareMathOperator{\coker}{coker}

% Category theory

\DeclareMathOperator*{\colim}{colim}
\DeclareMathOperator*{\Map}{Map}
\DeclareMathOperator*{\Maps}{Maps}

%\makeatletter
%\renewcommand\d[1]{\mspace{6mu}\mathrm{d}#1\@ifnextchar\d{\mspace{-3mu}}{}}
%\makeatother
%
%\newcommand{\naturalto}{%
%    \mathrel{\vbox{\offinterlineskip
%        \mathsurround=0pt
%        \ialign{\hfil##\hfil\cr
%            \normalfont\scalebox{1.2}{.}\cr
%                            %      \noalign{\kern-.05ex}
%        $\longrightarrow$\cr}
%    }}%
%}



\DeclareMathOperator{\DC}{DC} % differential characters
\DeclareMathOperator{\Del}{Del} % deligne cohomology
% \newcommand{\fsl}[1]{{\centernot{#1}}}
\renewcommand\d{\mathsf{D}}
\DeclareMathOperator{\Fun}{Fun}
\DeclareMathOperator{\Sing}{Sing}
\DeclareMathOperator{\Ob}{Ob}
\newcommand{\op}{\text{op}}

\DeclareMathOperator{\Tot}{Tot}
\DeclareMathOperator{\Nat}{Nat}
\DeclareMathOperator{\pr}{pr}
\DeclareMathOperator{\wt}{wt}
\DeclareMathOperator{\holim}{holim}
\DeclareMathOperator{\hocolim}{hocolim}
\newcommand{\desc}{\text{desc}}
\newcommand{\h}{\text{h}}
\newcommand{\pt}{\text{pt}}
\newcommand{\cyc}{\text{cyc}}


\DeclareMathOperator{\srep}{srep}
\DeclareMathOperator{\Emb}{Emb}




\usepackage{marginnote}

\begin{document}

\noindent\textbf{Claim:} the iterated integral map
\begin{equation*}
    \sigma: CC_*(\Omega^*(M)) \to \Omega^*(LM)
\end{equation*}
on differential forms is a map of graded vector spaces, but 
checking that it is a map of chain complexes uses the (graded) commutativity of 
differential forms. We will sketch this below.

The reason I want to emphasize this point is because
the analogous iterated integral map for (\v Cech-)Deligne cocycles will not be
a map of chain complexes. Since the Beilinson-Deligne cup product is commutative
up to homotopy, we will instead get $\sigma\circ b = d\circ\sigma + d\circ h$
for some term $h$. So we still obtain a map on cohomologies but not a map of 
chain complexes. This is problematic because the usual tools of spectral 
sequences etc rely on having a map of chain complexes (or perhaps this notion of 
up-to-exact-term chain map is enough?).

%The product structure on the Deligne complex is rather 
%subtle (it is not a dg algebra but a graded algebra in complexes $\oplus_p
%\Z_d(p)$) so this is just an educated guess for now. I will try to work it out 
%more carefully.

\hrulefill \\

Let $M$ be a (simply connected) manifold.
We denote by $CC_*(-)$ the Hochschild chain complex of a dg algebra.
We define a map
\begin{equation*}
    \sigma: CC_*(\Omega^*(M))=\bigoplus_{k=0}^\infty \Omega^*(M)^{\otimes k} \to 
    \Omega^*(LM)
\end{equation*}
as follows. Let
\begin{equation*}
    \omega_0\otimes \cdots \otimes \omega_k \in CC_k(\Omega^*(M)) = 
    \Omega^*(M)^{\otimes k}.
\end{equation*}
Using the projection maps $p_j: M^{k+1}\to M$, we obtain
\begin{equation*}
    p_0^*\omega_0 \wedge \cdots p_k^*\omega_k \in \Omega^*(M^{k+1}).
\end{equation*}
We transgress this form along the diagram:
\begin{equation*}
    \begin{tikzcd}
        \Delta^k \times LM \rar{ev_k}\dar{\pi} & M^{k+1} \\
        LM
    \end{tikzcd}
\end{equation*}
where
\begin{align*}
    ev_k : \Delta^k \times LM &\to M^{k+1} \\
    (t_1,\ldots, t_k, \ell) &\mapsto (\ell(0),\ell(t_1),\ldots,\ell(t_k)).
\end{align*}
So we obtain
\begin{equation*}
    \sigma(\omega_0\otimes\cdots\otimes\omega_k) =
    \pi_!ev_k^*(p_0^*\omega_0\wedge\cdots\wedge p_k^*\omega_k) \in \Omega^*(LM).
\end{equation*}
Under a suitable grading of the Hochschild complex (see Getzler-Jones-Petrack) 
this map extends to a map $\sigma$ of graded vector spaces.

Let us check that $\sigma$ is a map of chain complexes, i.e. that
\begin{equation}
    \sigma \circ b = d \circ \sigma,
    \label{chainmap}
\end{equation}
where
\begin{align}
    b(\omega_0\otimes \cdots \otimes \omega_k) &=
    \sum_{j=0}^k \pm \omega_0\otimes\cdots \otimes
    d\omega_j \otimes \cdots \omega_k \label{l1}\\
    &+ \sum_{j=0}^{k-1} \pm \omega_0\otimes \cdots\otimes
    (\omega_j\wedge\omega_{j+1}) \otimes \cdots \otimes \omega_k \label{l2} \\
    & \pm \omega_k\wedge\omega_0\otimes\omega_1\otimes \cdots\otimes 
    \omega_{k-1}\label{l3}
\end{align}
is the Hochschild differential. We can already see where commutativity of 
differential forms will be necessary: on the left hand side of 
eq.~\ref{chainmap} we have a term that looks like
$\sigma(\omega_k\wedge\omega_0\otimes \cdots)$ whereas on the right hand side of 
eq.~\ref{chainmap}, the 
differential forms will always be in order. 

More explicitly, using Stokes' theorem for $\pi_!$, we have
\begin{align*}
    d\sigma(\omega_0\otimes\cdots\otimes\omega_k) &= 
    d\pi_!ev_k^*(p_0^*\omega_0\wedge\cdots\wedge p_k^*\omega_k) \\
    &= \int_{\Delta^k}ev_k^*d(p_0^*\omega_0\wedge\cdots\wedge p_k^*\omega_k) \\
    &\pm\int_{\partial\Delta^k} ev_k^*(p_0^*\omega_0\wedge\cdots\wedge 
    p_k^*\omega_k)|_{\partial\Delta^k}.
\end{align*}
The first term, where the $d$ moves inside the integral, matches up with 
$\sigma$ applied to the term (\ref{l1}) in the Hochschild differential.
Similarly, by examining the behavior of $ev_k$ along each of the faces 
$\Delta^{k-1}\subset\Delta^k$ one can check that the integral over the boundary 
is equal to $\sigma$ applied to terms (\ref{l2}) and (\ref{l3}) of the 
Hochschild differential. The key point, however, is that in the integral along 
the last face ($t_k=0$) we need to commute the $p_k^*\omega_k$ all the way to
the left in order to obtain the term with $\omega_k\wedge\omega_0$ in 
(\ref{l3}). Hence we are crucially using the
strict (graded) commutativity of de Rham forms.


\end{document}

