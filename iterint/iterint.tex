
\documentclass{amsart}

\usepackage[colorlinks=true]{hyperref}
\usepackage{enumerate}
\usepackage{color}
\usepackage{tikz-cd}
\usepackage{amssymb}
% \usepackage{centernot}

\theoremstyle{plain}
\newtheorem{theorem}{Theorem}
\newtheorem{lemma}[theorem]{Lemma}
\newtheorem{proposition}[theorem]{Proposition}
\newtheorem{corollary}[theorem]{Corollary}

\theoremstyle{definition}
\newtheorem{definition}[theorem]{Definition}
\newtheorem{example}[theorem]{Example}
\newtheorem{exercise}[theorem]{Exercise}

\theoremstyle{remark}
\newtheorem{remark}[theorem]{Remark}
\newtheorem{question}[theorem]{Question}

% Fonts
\newcommand{\A}{\mathbb{A}}
\newcommand{\C}{\mathbb{C}}
\newcommand{\F}{\mathbb{F}}
\newcommand{\R}{\mathbb{R}}
\newcommand{\Q}{\mathbb{Q}}
\newcommand{\Z}{\mathbb{Z}}
\newcommand{\N}{\mathbb{N}}
\newcommand{\G}{\mathbb{G}}
\newcommand{\fr}{\mathfrak}

% Topology/geometry

\DeclareMathOperator{\Gr}{Gr}
\DeclareMathOperator{\Fl}{Fl}
\DeclareMathOperator{\PP}{\mathbb{P}}
\DeclareMathOperator{\Der}{Der}
\DeclareMathOperator{\Lie}{Lie}
\DeclareMathOperator{\SL}{SL}
\DeclareMathOperator{\GL}{GL}
\DeclareMathOperator{\SO}{SO}
\DeclareMathOperator{\UU}{U}
\DeclareMathOperator{\OO}{O}
\DeclareMathOperator{\Sp}{Sp}
\DeclareMathOperator{\HH}{H}
\DeclareMathOperator{\Symp}{Symp}
\DeclareMathOperator{\Spin}{Spin}
\DeclareMathOperator{\Pin}{Pin}
\DeclareMathOperator{\Td}{Td}
\DeclareMathOperator{\ind}{ind}
\DeclareMathOperator{\vect}{Vect}
\DeclareMathOperator{\Op}{Op}
\DeclareMathOperator{\ev}{ev}

% Representation theory

\DeclareMathOperator{\Ad}{Ad}
\DeclareMathOperator{\tr}{tr}
\DeclareMathOperator{\Str}{str}

% Algebra

\DeclareMathOperator{\End}{End}
\DeclareMathOperator{\Aut}{Aut}
\DeclareMathOperator{\Hom}{Hom}
\DeclareMathOperator{\sHom}{\mathscr{H}\!om}
\DeclareMathOperator{\sEnd}{\mathscr{E}\!nd}
\DeclareMathOperator{\id}{id}
\DeclareMathOperator{\irr}{irr}
\DeclareMathOperator{\Diff}{Diff}
\DeclareMathOperator{\gr}{gr}
\DeclareMathOperator{\im}{im}
\DeclareMathOperator{\ad}{ad}
\DeclareMathOperator{\rk}{rk}
\DeclareMathOperator{\Spec}{Spec}
\DeclareMathOperator{\RSpec}{RSpec}
\DeclareMathOperator{\Specm}{Specm}
\DeclareMathOperator{\Stab}{Stab}
\DeclareMathOperator{\Sym}{Sym}
\DeclareMathOperator{\Ext}{Ext}
\DeclareMathOperator{\ch}{ch}
\DeclareMathOperator{\cone}{cone}
\DeclareMathOperator{\cl}{Cl}
\DeclareMathOperator{\coker}{coker}

% Category theory

\DeclareMathOperator*{\colim}{colim}
\DeclareMathOperator*{\Map}{Map}
\DeclareMathOperator*{\Maps}{Maps}

%\makeatletter
%\renewcommand\d[1]{\mspace{6mu}\mathrm{d}#1\@ifnextchar\d{\mspace{-3mu}}{}}
%\makeatother
%
%\newcommand{\naturalto}{%
%    \mathrel{\vbox{\offinterlineskip
%        \mathsurround=0pt
%        \ialign{\hfil##\hfil\cr
%            \normalfont\scalebox{1.2}{.}\cr
%                            %      \noalign{\kern-.05ex}
%        $\longrightarrow$\cr}
%    }}%
%}



\DeclareMathOperator{\DC}{DC} % differential characters
\DeclareMathOperator{\Del}{Del} % deligne cohomology
% \newcommand{\fsl}[1]{{\centernot{#1}}}
\renewcommand\d{\mathsf{D}}
\DeclareMathOperator{\Fun}{Fun}
\DeclareMathOperator{\Sing}{Sing}
\DeclareMathOperator{\Ob}{Ob}
\newcommand{\op}{\text{op}}

\DeclareMathOperator{\Tot}{Tot}
\DeclareMathOperator{\Nat}{Nat}
\DeclareMathOperator{\pr}{pr}
\DeclareMathOperator{\wt}{wt}
\DeclareMathOperator{\holim}{holim}
\DeclareMathOperator{\hocolim}{hocolim}
\newcommand{\desc}{\text{desc}}
\newcommand{\h}{\text{h}}
\newcommand{\pt}{\text{pt}}
\newcommand{\cyc}{\text{cyc}}


\DeclareMathOperator{\srep}{srep}
\DeclareMathOperator{\Emb}{Emb}




\usepackage{marginnote}

\title{Notes on iterated integrals}

\author{Nilay Kumar}

\date{May 2019}

\begin{document}

\maketitle

\section{Classical construction}

\section{Factorization homology}

In this section we recast the construction of the iterated integral map in more
abstract language. The shift in perspective is in recognizing the Hochschild 
homology of a commutative dg algebra as the factorization homology of the
algebra over the circle,
\begin{equation*}
    HH_*(A) \simeq \int_{S^1} A.
\end{equation*}
For us, $A=\Omega^*(M)$ is the commutative algebra of differential forms on $M$
a simply-connected manifold $M$. We construct a map
\begin{equation*}
    f \colon \int_{S^1}\Omega^*(M) \longrightarrow \Omega^*(LM)
\end{equation*}
using the definition of factorization homology.

Let $A\colon \mathsf{Disk}_n\to \mathsf{Ch}$ be an $n$-disk algebra in cochain
complexes of abelian groups. Then the factorization homology of $A$ over an
$n$-manifold $X$ is defined as a homotopy colimit
\begin{equation*}
    \int_X A = \hocolim\left(\mathsf{Disk}_{n/X} \xrightarrow{\mathsf{fgt}}
    \mathsf{Disk}_n \xrightarrow{A} \mathsf{Ch}\right).
\end{equation*}
Intuitively this definition tells us to factorize $X$ into disjoint disks
labeled with
elements of $A$ and to average over all such labeled factorizations, allowing 
disks to collide via the multiplication on $A$. By definition, then, a map from
the factorization homology to a cochain complex $C$ is a (homotopy) cocone
$\eta\colon A\circ\mathsf{fgt} \to \underline{C}$, i.e. a (homotopy) natural
transformation to the constant functor on $C$.

In our setup, then, to construct the map $f$ it is enough to give a (homotopy) 
natural transformation
\begin{equation*}
    \eta \in \Fun(\mathsf{Disk}_{1/S^1}, \mathsf{Ch})(A\circ\mathsf{fgt}, 
    \underline{\Omega^*(LM)}).
\end{equation*}
This cocone is constructed as follows. Consider the composite
\begin{equation*}
    \mathsf{Disk}_{1/S^1} \longrightarrow ({}_{LM\backslash}\mathsf{Mfld})^\op
    \longrightarrow \mathsf{CAlg}_{/\Omega^*(LM)}
\end{equation*}
where the first arrow is given by applying $\Maps_{\mathsf{Mfld}}(-,M)$,
\begin{equation*}
    \left(\sqcup_I \R \hookrightarrow S^1\right) \mapsto \left(LM \to 
    \Maps_{\mathsf{Mfld}}(\sqcup_I\R, M)\right)
\end{equation*}
and the second arrow is given
\begin{equation*}
    \left(LM \to X\right) \mapsto \left(\Omega^*(X)\to\Omega^*(LM)\right).
\end{equation*}
The composite is therefore given
\begin{equation*}
    \left(\sqcup_I \R\hookrightarrow S^1\right) \mapsto 
    \left(\Omega^*(\Maps(\sqcup_I\R, M)) \to \Omega^*(LM)\right).
\end{equation*}
Now as functors to an overcategory are the same as functors from the right cone, 
we obtain a functor $\mathsf{Disk}_{1/S^1}^\triangleright \to \mathsf{Ch}$ 
sending the cone object $\ast$ to $\Omega^*(LM)$. As the source category
is isomorphic to $\mathsf{Disk}_{1/S^1}$, we obtain the desired 
cocone.\marginnote{Work this out}

\appendix

\section{Differential forms on free loop spaces}

results about forms on the presheaf versus forms on the Fr\'echet space


\end{document}

