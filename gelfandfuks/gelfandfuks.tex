\documentclass{amsart}

\usepackage{amsmath}
\usepackage{amsthm}
\usepackage{amssymb}
\usepackage{tikz-cd}
\usepackage{hyperref}
\usepackage{enumitem}

\theoremstyle{definition}
\newtheorem{lemma}{Lemma}
\newtheorem{proposition}[lemma]{Proposition}
\newtheorem{definition}[lemma]{Definition}
\newcommand{\im}{\text{im}}
\newcommand{\intmul}{\lrcorner\;}
\newcommand{\mul}{\text{mul}}
\newcommand{\R}{\mathbb{R}}
\newcommand{\Vect}{\text{Vect}}
\newcommand{\Sym}{\text{Sym}}
\newcommand{\holim}{\text{holim}}

\begin{document}

\title{Notes on Hennion-Kapranov}
\date{Summer 2019}

\maketitle

\vspace{.5cm}

In this note we detail the computation of the (continuous) Lie algebra 
cohomology of the Lie algebra of vector fields on a smooth manifold, due 
originally to Gelfand and Fuks. We follow the exposition of Hennion and 
Kapranov, which uses the local-to-global tools of factorization homology to 
compute this cohomology.

Let us outline the idea behind the computation. Let $M$ be a smooth manifold.
\begin{enumerate}[label=(\roman*)]
    \item For $U\subset M$ we may consider the cdga
        \begin{equation*}
            \mathcal{A}(U) := \mathsf{CE}^*(\Vect(U))
        \end{equation*}
        computing the cohomology of vector fields on $U$. The assignment 
        $U\mapsto \mathcal{A}(U)$ then defines a factorization algebra 
        $\mathcal{A}$ on $M$, whose global sections (factorization homology) is 
        the object we would like to calculate. Peculiar to the case of cdgas, 
        factorization algebras are just (homotopy) cosheaves. Furthermore, the 
        cosheaf $\mathcal{A}$ is actually locally constant.
    \item According to Gelfand and Fuks,
        the Lie algebra cohomology of vector fields can locally be written in 
        terms of the (Thom-Whitney) cochains of a particular CW complex $Y$. It 
        follows that the cosheaf $\mathcal{A}$ on $M$ is weakly equivalent to a 
        certain locally constant cosheaf $\mathcal{Y}$ on $M$ associated to $Y$.
    \item It now remains to compute the global sections of $\mathcal{Y}$. For 
        this we use non-abelian Poincar\'e duality, incarnated in this context 
        as a close relationship between locally constant sheaves and cosheaves,
        which describes the factorization homology $\mathcal{Y}(M)$ as the space 
        of sections of a certain fibration over $M$ with fiber $Y$.
\end{enumerate}

\section{Gelfand-Fuks theory}


\section{Globalization}

%To globalize the local theory developed above, Hennion and Kapranov use the 
%machinery of factorization algebras of cdgas.
%\begin{definition}
%    Let $M$ be a smooth manifold and $(\mathcal{C}, \otimes, 1)$ be a symmetric 
%    monoidal category. A \textbf{prefactorization algebra} $\mathcal{A}$ on $M$ 
%    with values in $\mathcal{C}$ consists of the following data:
%    \begin{enumerate}
%        \item an assignment of an object $\mathcal{A}(U)\in \mathcal{C}$ to each 
%            open set $U\subset M$, with $A(\varnothing)=1$;
%        \item 
%    \end{enumerate}
%\end{definition}

% TODO compare HK's direct approach with the approach of Lurie in HA
% TODO understand the relationship between lcfas and diskn algebras




\appendix

\section{The algebraic picture}

In the above we have considered smooth vector fields on smooth manifolds. 
One might ask, however, whether an analog of the Gelfand-Fuks result holds in 
the setting of (smooth) algebraic varieties. Hennion and Kapranov note that this 
question dates back to B.L. Feigin in the 80's, but there was apparently no 
progress on the question, even for the case of curves. The bulk of their paper 
is concerned with determining the Lie algebra cohomology of regular vector 
fields in this case. I am not very familiar with algebraic geometry, which is 
why I decided to focus on the smooth case. I think it is worth outlining the 
constructions, though, to get a sense of the ideas involved.

The overall argument is similar to the one described above. For $X$ a smooth 
affine algebraic variety over $\C$
we wish to determine the Chevalley-Eilenberg complex of
the (right derived functor of the) global sections of the tangent sheaf 
$\mathsf{CE}^\ast(R\Gamma(X, \mathcal{T}_X))$. Let us write 
$\mathfrak{l}=R\Gamma(X, \mathcal{T}_X)$ in what follows.

We start by representing the Lie algebra homology (tor, not ext) of 
$\mathfrak{l}$ as the factorization homology of, roughly, a certain 
$\mathcal{D}$-module on the Ran space $\text{Ran } X$. What do we mean by 
this? Recall that the Ran space of $X$ is the set of all nonempty finite 
subsets of $X$, topologized to allow points to collide. In an 
appropriately general category of algebraic objects, the Ran space can 
be viewed as the colimit of a certain diagram $X^\mathcal{S}$ of powers 
of $X$. Here $\mathcal{S}$ is the category of nonempty finite sets and 
surjections. For $I\twoheadrightarrow J$ the map $X^J\to X^I$ is given 
by the diagonal embedding corresponding to $I\twoheadrightarrow J$.

Now a \textbf{lax $\mathcal{D}^!$-module} on $X^\mathcal{S}$ is, roughly, the 
assignment to each variety in the diagram $X^\mathcal{S}$ an object of the 
derived category of quasicoherent right $\mathcal{D}$-modules in a suitably 
compatible way. Intuitively, since the Ran space is a colimit, its category 
of $\mathcal{D}$-modules must be a limit of categories of $\mathcal{D}$-modules 
on powers of $X$. The word lax is signifying that the compatibility condition 
can be strengthened to give a notion of strict $\mathcal{D}^!$-module.

Write $\mathcal{L}=L\otimes_{\mathcal{O}_X}\mathcal{D}_X$, where 
for us, $L=\mathcal{T}_X$.
Then we can define a 
lax $\mathcal{D}^!$-module $\mathcal{C}_1$ on $X^\mathcal{S}$ by
\begin{equation*}
    \mathcal{C}_{1}^{(I)} = (\delta_I)_* \mathcal{L},
\end{equation*}
where $\delta_I: X\to X^I$ is the diagonal embedding.
Given a surjection $g:I\twoheadrightarrow J$ the corresponding structure map
\begin{equation*}
    \delta_g: (\delta_g)_*(\delta_J)_*\mathcal{L} \to (\delta_I)_*\mathcal{L}
\end{equation*}
is the isomorphism coming from the equality $\delta_g\circ\delta_J=\delta_I$.

There is a symmetric monoidal product on the category of lax 
$\mathcal{D}^!$-modules, denoted by $\otimes^*$ (distinct from the ``chiral'' 
tensor product). It is defined as follows:
\begin{equation*}
    (\mathcal{E}\otimes \mathcal{F})^{(I)} = \bigoplus_{I_1\cup I_2=I}
    \mathcal{E}^{(I_1)} \boxtimes \mathcal{F}^{(I_2)}
\end{equation*}
In our case, $\mathcal{C}_1$ becomes a Lie algebra object with respect to this 
tensor product. Thus we can define
\begin{equation*}
    \mathcal{C}_* = (\Sym^{*\geq 1}_{\otimes^*} \mathcal{C}_1[1], 
    d_{\text{CE}}).
\end{equation*}
The factorization homology of $\mathcal{C}_*$, which we now define, will compute 
the Chevalley-Eilenberg homology of $\mathfrak{l}$.

The factorization homology of a lax $\mathcal{D}^!$-module $\mathcal{E}$ on $X$ 
is
\begin{equation*}
    \int_X \mathcal{E} = \holim_{I\in\mathcal{S}}
    R\Gamma_{\text{dR}}(X^I, \mathcal{E}^{(I)}) \in 
    \mathsf{Ind}(\mathsf{Perf})=\mathsf{Ch},
\end{equation*}
i.e. the global sections of $\mathcal{E}$ on the Ran space. A few words about 
$R\Gamma_{\text{dR}}$ are in order: this is the global sections functor applied 
to the de Rham complex of a $\mathcal{D}$-module. Recall that the de Rham 
complex is $\text{dR}(\mathcal{M})=\mathcal{M} 
\otimes^h_{\mathcal{D}_X}\mathcal{O}_X$. With this in mind, Kapranov and Hennion 
show that
\begin{equation*}
    \int_X \mathcal{C}_* \simeq \text{CE}_*(\mathfrak{l}).
\end{equation*}
From this, invoking duality, they obtain
\begin{equation}
    \text{CE}^*(\mathfrak{l}) \simeq 
    R\Gamma_{\text{dR}}^{[[c]]}(X^{\mathcal{S}}, \psi(\mathcal{C}^\vee)).
    \label{ce}
\end{equation}
Here we are taking the compactly supported sections of a certain Verdier dual of 
$\mathcal{C}$. This maneuver is the analog of non-abelian Poincar\'e duality 
above and the close relationship between sheaves and cosheaves.

Now $\psi(\mathcal{C}^\vee)$ satisfies some nice properties, giving us 
a factorization algebra (in the usual sense) on $X_{\text{an}}$,
\begin{equation*}
    \mathcal{A}(U) = \holim_{I\in\mathcal{S}} R\Gamma_c(U^I, 
    \text{dR}(\psi(\mathcal{C}^\vee)^{(I)})_{\text{an}}),
\end{equation*}
such that its global sections, its factorization homology, on $X_\text{an}$ is 
the right-hand side of \ref{ce}. Thus we obtain an identification
\begin{equation*}
    \int_{X_\text{an}} \mathcal{A} \simeq \text{CE}^*(\mathfrak{l}).
\end{equation*}
The remainder of the calculation is now similar to the local argument given in 
the topological case, showing that the factorization algebra is locally given as 
a section space, from which the global identification as a certain section space 
follows.











\end{document}
